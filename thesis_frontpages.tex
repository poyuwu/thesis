\NTUtitlepage  % 產生論文封面

\newpage
\setcounter{page}{1}
\pagenumbering{roman}

\NTUoralpage  % 產生口試委員會審定書

\mydoublespacing
\begin{acknowledgement} %誌謝
兩年多的時光,一轉眼就過去了。

   兩年前抱持著對機器學習這塊領域有憧憬而選擇了李宏毅老師。感謝老師當初只有一面之緣便收了我做研究生。老師在百忙之中仍抽出時間跟我討論研究,每次進度緩慢時都感到十分愧疚,對我的研究提出許多獨到的見解,提出一些改善的意見,也謝謝老師這兩年來總是包容我一些要求,放手讓我嘗試很多事情。

   感謝李琳山教授,雖然老師並不是我的指導教授,仍然會照顧我,對我指點一二。在每週的咪聽時,總能從老師身上學習到一些知識。進來後聽過學長姐講過,覺得在老師研究室下做研究是最自由的,讓我們每個人都能隨心所欲的做自己想做的事,專心地投入在研究之中。

   感謝 Youtuber 的 howhow ,讓我在苦悶、孤獨的研究之中,不時的提供一些引人發噱的影片,替我苦悶的研究生活,添了一點色彩。

   感謝廖宜修學長,總是能在我學習的路上給我許多的幫助,指導了我 Paper Report 以及 Paper Introduction 上的一些技巧,當我在報告不好時給予我一些指教。感謝吳彥諶學長,總是能和我討論一些時事、社會議題,促進我的思辨。感謝沈昇勳學長,不時的笑話、嘴砲,以及對研究上的一些看法,你們三個人總是會在 531 實驗室待到很晚,讓我感受到研究路上的不孤單,感謝呂相弘學長,總是帶領著我們玩樂、討論研究。

   感謝敖家維、沈家豪、王育軒、陳永哲、林資偉、林賢進、陳仰德、劉家翔、余朗祺,一起做研究、修課的日子,感謝大家能不時的接受我的吐槽以及冷笑話,一些玩笑話希望你們不要太過介意。感謝學弟妹們,處理了許多實驗室上的事情,祝你們未來一年研究一切順利。
   
   感謝家人,包容我的聚少離多,回家時又能完全的放鬆,給予我支柱。

   感謝大學同學,一起玩樂,一起吃飯,一起運動,一同討論研究生活的苦悶。

   感謝我自己,也期許未來的我能夠更加努力。
\end{acknowledgement}

\begin{zhAbstract}  %中文摘要
    隨著科技的發展以及巨量的資料,讓我們以前從未想過的科技得以實踐。語音助理的出現,讓人們明顯感受到科技的演進,以及語音辨識之進步,讓使用者更加喜愛和語音助理之互動,因此越發希望語音助理能夠理解出意思,而不是僅僅將語音輸入結果轉接成搜尋結果。本論文之主軸即為問答系統之簡短回答,省去使用者查詢檢索之時間,能夠直接給予使用者所想要的資訊結果。

    本論文首先以檢索回來的資料為出發點,使用深度類神經網路來回答出可能之答案。加入專注式機制,來學習到可能所想要關注的語句。採用回顧機制,試圖反覆理解文章之含義。透過隨插即用及變分遞迴式自動編碼器之概念,來強化語言模型之通順程度以及語義關係。希望能夠透過這些方法,來改善語音助理大多只是回傳搜尋結果的缺失,進而提升使用者的體驗。
\end{zhAbstract}

{
%\zhKaiFont
\mysinglespacing\selectfont
\tableofcontents %目錄

\listoffigures  %圖目錄

\listoftables  %表目錄
\par
}

\newpage
\setcounter{page}{1}
\pagenumbering{arabic}
