\NTUtitlepage  % 產生論文封面

\newpage
\setcounter{page}{1}
\pagenumbering{roman}

\NTUoralpage  % 產生口試委員會審定書

\mydoublespacing
\begin{acknowledgement} %誌謝
    To Be Continued
\end{acknowledgement}

\begin{zhAbstract}  %中文摘要
    隨著科技的發展以及巨量的資料,讓我們以前從未想過的科技得以實踐。語音助理的出現,讓人們明顯感受到科技的演進,以及語音辨識之進步,讓使用者更加喜愛和語音助理之互動,因此越發希望語音助理能夠理解出意思,而不是僅僅將語音輸入結果轉接成搜尋結果。本論文之主軸即為問答系統之簡短回答,省去使用者查詢檢索之時間,能夠直接給予使用者所想要的資訊結果。

    本論文首先以檢索回來的資料為出發點,使用深度類神經網路來回答出可能之答案。加入專注式機制,來學習到可能所想要關注的語句。採用回顧機制,試圖反覆理解文章之含義。透過隨插即用及變分遞迴式自動編碼器之概念,來強化語言模型之通順程度以及語義關係。希望能夠透過這些方法,來改善語音助理大多只是回傳搜尋結果的缺失,進而提升使用者的體驗。
\end{zhAbstract}

{
%\zhKaiFont
\mysinglespacing\selectfont
\tableofcontents %目錄

\listoffigures  %圖目錄

\listoftables  %表目錄
\par
}

\newpage
\setcounter{page}{1}
\pagenumbering{arabic}
