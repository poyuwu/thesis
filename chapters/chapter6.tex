\section{結論}
本碩論主要目的是要透過閱讀文章以及問句,來產生出相對應的答案。過去的問答系統有的是選項式的問答,此種我們並無法直截了當的知道機器究竟學習到的內容。另外有些是採用大量的自然語言處理分析,抽取出文章內可能句子的段落,因此若是答案並非直接在文章內部,可能無法有效的抓取出答案。在此,透過類神經網路的方法,試著去抓取語義上的意思,以及文章當中可能的重點,藉此來獲取潛在的意思。而在機器閱讀理解數據集的排行榜下,雖然我們並未達到最高分數,但其模型大都為文本提取方式,從文本之中選出取可能的範圍,因此僅只能擷取字面上的意思,對於是非的 yes、no 問句這種需要理解後得到的答案則會相對難產生,然而其優點就是相對產生的文句會符合人們所使用的文句。

第三章我們提出一個能夠解決特定小範圍的領域知識的模型,利用了遞迴式網路以及專注式機制,來產生出答案,此模型幾乎沒有使用傳統自然語言處理的方法,僅僅讓文章以句子層級來表示。

第四章我們試圖使用相同的模型架構,並轉換為分類問題,雖然結果並不如預期,但也另類的證明了第三章的架構是能成功的。

第五章引入了隨插即用的概念,以變分遞迴式自動編碼器當作預先訓練之模型,使得較長的句子更能夠完整產生。%雖說隨插即用是建立在生成對抗模型,但同為生成模型,%變分自動編碼器與生成對抗網路%同為生成模型(Generative model),在生成對抗網路這種

%句子層級
%\itemsep -2pt
%\begin{itemize}
%\end{itemize}
%ROUGE 不一定好 abstract QA
%TODO
\section{未來展望}
本論文提出解決問答系統的一種方法,雖成果仍然不到人類水平,但已經勉強可以讓機器學習到該專注在哪些句子之上,且能夠學習到詞彙的意思,進而回答,若是能結合文句擷取及此方法,加以整合,或許能夠有更好的結果。

再者,以 ROUGE 為衡量之方法也僅僅是現階段在問答系統中,較好的衡量方法,基本上 ROUGE 無法考慮字義上的意思來衡量,僅只是靠 n 元語法來加以計算分數。

最後,由於我們的訓練方法是降低每個字的交叉熵以進行,然而測量卻是使用 ROUGE 來衡量,然而近期有許多加強式學習(Reinforcement Learning)的方法,可以針對 ROUGE 值進行訓練,或許可以獲得更好的結果。
%Prove siri
%加入RL技術,objective function: ROUGE,Ensemble,
%TODO
