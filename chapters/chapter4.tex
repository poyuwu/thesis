\section{簡介}
接續前一章所提到,我們需要使用一個過濾器,從檢索的回傳值中挑選出最可能含有答案的文本。困難的點在,這並非僅僅是找尋文本中是否包含有查詢詞,因為這在檢索時已經使用過了。更深入的是,我們需要能夠判別語意是否能夠回答到查詢詞。故在此若僅只是使用詞頻(Term Frequency,tf)- 逆向檔案頻率(Inverse Document Frequency,idf),對於要選擇出是否包含著答案的文本並不容易。對此,我們使用了相同的模型來解決這件事情,差別僅僅是把編碼器編碼後的結果,通過一層分類器來判別文本是否有包含查詢詞,如圖(\ref{fig:classifier})%TODO

\section{基本實驗配置}
在本章節,我們依然提出兩個基準實驗模型,如圖 \ref{} 、\ref{} 。
%TODO introducing CNN
\section{實驗結果與討論}
%TODO MAP
%TODO ROUGE
\section{本章總結}

